%!TEX root=./optim_report.tex
\section{RK2 - Ralston Method}

Here we present the first Runge-Kutta Method, a $2nd$ order method also known as RK2-Ralston, which we refer to as RK2 in the experiments.

\begin{algorithmic}
\STATE Given $x_0$
\FOR {$t $ in $[0,T]$ do}
  \STATE $k_1 \gets \nabla f(x_t)$
  \STATE $k_2 \gets \nabla f(x_t - \frac{2\alpha}{3} k_1) $
  \STATE $x_{t+1} \gets x_n - \frac{\alpha}{4}(k_1 + 3 k_2)$
\ENDFOR
\end{algorithmic}

\subsection{Main Results}
\begin{thm} Let $f(x) \in C_{\beta}^{2,2}( \mathbb{R}^n) \cap  C_{\beta}^{2,1}( \mathbb{R}^n)$ and $f$ is bounded below, then the RK2-Ralston Method gap between $x_t$ and some local minima $x^*$ is given by, where $\alpha = \frac{2}{\beta}$ :
\begin{align*}
f(x_t) - f(x^*) \leq \frac{4}{3 \beta} \frac{ || x_1 - x^* ||_2^2}{t-1}
\end{align*}
\end{thm}

To prove the above proposition we need the following lemma, where we show the amount of progress made by our integration scheme in $1$ step.
\begin{lemma}
Let $f : \mathbb{R}^d \rightarrow \mathbb{R} \in C_{\beta}^{2,2}( \mathbb{R}^n) \cap  C_{\beta}^{2,1}(\mathbb{R}^n)$ and $f$ be bounded below. Let $\Delta x =  \frac{1}{2\beta}(k_1 + 3k_2)$, then  we show that,
\begin{align}
f(x - \Delta x ) - f(x) \leq - \frac{3}{4 \beta} || \nabla f(x) ||^2
\end{align}
\end{lemma}

%!TEX root=./optim_report.tex

\begin{proof}(Lemma 1)
Let $\Delta x =  \frac{1}{2\beta}(k_1 + 3k_2)$, where $\alpha = \frac{2}{\beta}$ then
\begin{equation}
\begin{aligned}\label{ineq0}
f(x - \Delta x) - f(x) &\leq \nabla f(x)^T (x - \Delta x - x) + \frac{\beta}{2} || x - x - \Delta x ||^2 \\
&= -  \nabla f(x)^T ( \Delta x) + \frac{1}{16 \beta} || \Delta x ||^2 \\
&= -\frac{1}{2 \beta} \nabla  f(x)^T ( k_1 + 3k_2) + \frac{1}{16 \beta} || \Delta x ||^2 \\
&= -\frac{1}{2\beta}\nabla f(x)^T k_1 - \frac{3}{2\beta}\nabla f(x)^T k_2 + \frac{1}{16 \beta} || k_1 ||_2^2 + \frac{9}{16 \beta} || k_2 ||^2_2 + \frac{6}{16\beta}\left\langle k_1, k_2 \right\rangle \\
&= -\frac{1}{2\beta}\nabla f(x)^T k_1 +  \frac{1}{16 \beta} || k_1 ||_2^2 -  \frac{3}{2\beta}\nabla f(x)^T k_2 +  \frac{9}{16 \beta} || k_2 ||^2_2  + \frac{6}{16\beta}\left\langle k_1, k_2 \right\rangle \\
&= -\frac{7}{16 \beta}|| k_1 ||_2^2 -\frac{1}{16 \beta}k_2^T(24 \nabla f(x) - 9 k_2) + \frac{6}{16\beta}\left\langle k_1, k_2 \right\rangle \\
&= -\frac{7}{16 \beta}|| k_1 ||_2^2 - \frac{24}{16 \beta}k_2^Tk_1 + \frac{6}{16 \beta}\left\langle k_1, k_2 \right\rangle + \frac{9}{16\beta}|| k_2 ||_2^2 \\
&= -\frac{7}{16 \beta}|| k_1 ||_2^2 - \frac{18}{16 \beta} \left\langle k_1, k_2 \right\rangle + \frac{9}{16 \beta}|| k_2 ||_2^2
\end{aligned}
\end{equation}

Now, using a Taylor Series approximation for $\nabla f \big( x -  \frac{2\alpha}{3}k_1 \big)$, we get that,
\begin{equation}
\begin{aligned}\label{ineq1}
\nabla f \big( x - \frac{2\alpha}{3}k_1 \big) &= \nabla f(x) - \frac{2\alpha}{3} \nabla^2 f(x) \nabla f(x) + \mathcal{O}( |\frac{2\alpha}{3} |^2 ) \\
\implies  k_2^Tk_1 &= \nabla f \big( x - \frac{2\alpha}{3}k_1 \big)^T\nabla f(x) \\
 &= \nabla f \big( x - \frac{2\alpha}{3}\nabla f(x) \big)^T \nabla f(x) \\
&= || \nabla f(x) ||_2^2 - \frac{2\alpha}{3} \nabla f(x)^T \nabla^2 f(x) \nabla f(x)
\end{aligned}
\end{equation}
And, using \eqref{ineq1},
\begin{equation}
\begin{aligned} \label{ineq2}
|| k_2 ||_2^2 &= ||  \nabla f(x) - \frac{2\alpha}{3} \nabla^2 f(x) \nabla f(x)  ||_2^2   \\
&= || \nabla f(x)||_2^2 + \frac{4\alpha}{9}||\nabla^2 f(x) \nabla f(x)  ||_2^2 - \frac{4\alpha}{3} \nabla f(x)^T \nabla^2 f(x) \nabla f(x)
\end{aligned}
\end{equation}
Hence, using \eqref{ineq1} and \eqref{ineq2}
\begin{align*}
f(x - \Delta x) - f(x) & \leq  -\frac{7}{16 \beta}|| k_1 ||_2^2 - \frac{18}{16 \beta} \left\langle k_1, k_2 \right\rangle + \frac{9}{16 \beta}|| k_2 ||_2^2 \\
&= -\frac{7}{16 \beta}|| \nabla f(x) ||_2^2 -  \frac{18}{16 \beta} || \nabla f(x) ||_2^2 + \frac{12}{16 \beta^2} \nabla f(x)^T \nabla^2 f(x) \nabla f(x) \\
&+  \frac{9}{16\beta}( || \nabla f(x)||_2^2 + \frac{4}{9\beta}||\nabla^2 f(x) \nabla f(x)  ||_2^2 - \frac{4}{3\beta} \nabla f(x)^T \nabla^2 f(x) \nabla f(x) )  \\
&= -\frac{16}{16 \beta}|| \nabla f(x)||_2^2 + \frac{1}{4 \beta^2}||\nabla^2 f(x) \nabla f(x)  ||_2^2 \\
&= -\frac{1}{\beta} || \nabla f(x)||_2^2 + \frac{1}{4 \beta^2}||\nabla^2 f(x) \nabla f(x)  ||_2^2 \\
\end{align*}
Using the lipschitz property of the Hessian of $f$, $||\nabla^2 f(x) u - \nabla^2 f(x) v||_2^2 \leq \beta || u-v ||_2^2 $, we get that,
\begin{equation}
\begin{aligned}
f(x - \Delta x) - f(x) & \leq -\frac{1}{\beta} || \nabla f(x)||_2^2 + \frac{1}{4\beta^2}||\nabla^2 f(x) \nabla f(x)  ||_2^2 \\
& \leq -\frac{4}{4\beta}|| \nabla f(x) ||_2^2 + \frac{\beta}{4 \beta^2}|| \nabla f(x) ||_2^2 \\
& = -\big( \frac{4}{4\beta} - \frac{\beta}{8\beta^2}   \big)  || \nabla f(x) ||_2^2  \\
&= -\frac{3}{4\beta} || \nabla f(x) ||_2^2
\end{aligned}
\end{equation}
\end{proof}
\begin{proof}(Theorem 1)
Using Lemma 1, we have $f(x_{t+1}) - f(x) \leq -\frac{3}{4\beta}|| \nabla f(x) ||_2^2 $. Now, let $\delta_t = f(x_t) - f(x^*)$, then note that:
\begin{align*}
\delta_{t+1} \leq \delta_t - \frac{3}{4\beta}|| \nabla f(x) ||_2^2
\end{align*}
Now, by convexity of $f(x)$ we have:
\begin{align}
\delta_t &\leq \nabla f(x_t)^T (x_t - x^*) \\
 &\leq || x_t - x^* ||_2 * || \nabla f(x_t) ||_2 \\
\frac{1}{|| x_t - x^* || }\delta_t^2 & \leq  || \nabla f(x_t) ||_2^2
\end{align}

Now, note that $|| x_t - x^*||_2^2$ is decreasing, using the following inequality
\begin{align*}
\big( \nabla f(x) - \nabla f(y)  \big)^T(x-y)  \geq \frac{1}{\beta} || \nabla f(x) - \nabla f(y) ||_2^2
\end{align*}
Using the above and the fact that $\nabla f(x^*) = 0$,
\begin{align*}
|| x_{t+1} - x^* ||_2^2 &= || x_t - \Delta x_t - x^* ||_2^2 \\
&= || x_t - x^* ||_2^2 + || \Delta x_t ||_2^2 - 2 \Delta x_t^T(x_t - x^*) \\
&= || x_t - x^* ||_2^2 - \frac{1}{\beta}(k_1 + 3k_2)^T (x_t - x^*) + \frac{1}{4 \beta^2}|| k_1 + 3k_2 ||_2^2 \\
&= || x_t - x^* ||_2^2 - \frac{1}{ \beta}k_1^T (x_t - x^*) + \frac{1}{4 \beta^2}|| k_1 ||_2^2  \\
& \quad \quad \quad - \frac{3}{\beta}k_2^T (x_t - x^*) + \frac{9}{4 \beta^2}|| k_2 ||_2^2 + \frac{6}{4 \beta^2} k_1 ^T k_2 \\
&= || x_t - x^* ||_2^2 - \frac{4}{4 \beta^2}||k_1 ||_2^2 + \frac{1}{4 \beta^2}|| k_1 ||_2^2  \\
& \quad \quad \quad - \frac{12}{4 \beta^2}|k_2||_2^2 + \frac{9}{4 \beta^2}|| k_2 ||_2^2 + \frac{6}{4 \beta^2} k_1 ^T k_2 \\
&= || x_t - x^* ||_2^2 - \frac{3}{4 \beta^2}||k_1||_2^2 - \frac{3}{4 \beta^2}||k_2||_2^2 + \frac{6}{4 \beta^2} k_1 ^T k_2 \\
&= || x_t - x^* ||_2^2 -  \frac{3}{4 \beta^2}|| k_1 - k_2||_2^2  \\
& \leq || x_t - x^* ||_2^2
\end{align*}
We will show that,
\begin{align}
\delta_{t+1} \leq \delta_t - \frac{3}{4 \beta || x_1 - x^* ||_2^2} \delta_t^2
\end{align}
Now, let $\omega = \frac{3}{4 \beta   || x_1 - x^* ||_2^2}$, %then note that: (Proof in Bubek page - 269)
\begin{align*}
& \frac{1}{\delta_t} \geq \omega (t-1) \\
\implies & f(x_t) - f(x^*) \leq \frac{4}{3 \beta} \frac{ || x_1 - x^* ||_2^2}{t-1} \xrightarrow{t \to \infty} 0
\end{align*}
\end{proof}


