%!TEX root=./optim_report.tex
\section{RK2 - Ralston Method}

Here we present the first Runge-Kutta Method, a $2nd$ order method also known as RK2-Ralston, which we refer to as RK2 in the experiments.

\begin{algorithmic}
\STATE Given $x_0$
\FOR {$t $ in $[0,T]$ do}
  \STATE $k_1 \gets \nabla f(x_t)$
  \STATE $k_2 \gets \nabla f(x_t - \frac{2\alpha}{3} k_1) $
  \STATE $x_{t+1} \gets x_n - \frac{\alpha}{4}(k_1 + 3 k_2)$
\ENDFOR
\end{algorithmic}

\subsection{Main Results}
\begin{thm} Let $f(x) \in C_{\beta}^{2,2}( \mathbb{R}^n) \cap  C_{\beta}^{2,1}( \mathbb{R}^n)$ and $f$ is bounded below, then the RK2-Ralston Method gap between $x_t$ and some local minima $x^*$ is given by, where $\alpha = \frac{2}{\beta}$ :
\begin{align*}
f(x_t) - f(x^*) \leq \frac{4}{3 \beta} \frac{ || x_1 - x^* ||_2^2}{t-1}
\end{align*}
\end{thm}

To prove the above proposition we need the following lemma, where we show the amount of progress made by our integration scheme in $1$ step.
\begin{lemma}
Let $f : \mathbb{R}^d \rightarrow \mathbb{R} \in C_{\beta}^{2,2}( \mathbb{R}^n) \cap  C_{\beta}^{2,1}(\mathbb{R}^n)$ and $f$ be bounded below. Let $\Delta x =  \frac{1}{2\beta}(k_1 + 3k_2)$, then  we show that,
\begin{align}
f(x - \Delta x ) - f(x) \leq - \frac{3}{4 \beta} || \nabla f(x) ||^2
\end{align}
\end{lemma}

\input{proof_rk2_ralston}
